% !TEX root = /home/frank/School/thesis_text/thesis.tex


\chapter*{Abstract}

There are many applications that could benefit from increased computing performance. In the past this need was mostly satisfied by Moore's law; this this enabled hardware designers and programmers to continue supporting a sequential programming paradigm. Due to power constraints it proved unmaintainable to continue increasing the frequency. The need for processing power kept rising and Moore's law is still valid, so parallelism was introduced to enable an increase in performance. The CPU's used in desktop and server computers have become multi-core processors and alternatives such as General Purpose Graphics Processing Units and Field Programmable Gate Arrays can offer extreme parallelism for applications that can benefit from it. The goal of this thesis is to study a new type of heterogeneous architecture, the Zynq SoC from Xilinx. This chip is a combination of a dual core ARM processor and an FPGA on the same die. Ideally this brings the best features of both processors to the table: the ease of use and floating point performance of the CPU and the massively parallel performance of the FPGA. In the first chapter an introduction is given to the current state of parallel computing. A summary is given of the different types of computing components, as well as an introduction to a system for determining which processing unit is most fit to run a certain algorithm. The second chapter gives an overview of the features of the Zynq platform, discussing both the processing system, the programmable logic as well as the interconnection between these two. The toolchain used to develop for this platform is introduced, as well as the video processing system used to perform the performance analysis. This system, the Zynq Base Targeted Reference Design is a real-time video processing system which has stringent throughput requirements. The complete stack is discussed: the hardware implementation, the operating system and the user application. In the third chapter details High Level Synthesis, a technique for converting high level programming languages into hardware. The many features of the tool used for this thesis, Vivado HLS, are also discussed in this chapter. In the fourth chapter the factors influencing the performance are discussed, as well as the roofline model, a way to visually represent the performance of a system. Concepts from this roofline model are used to interpret the performance of the system.
% !TEX root = /home/frank/School/thesis_text/thesis.tex


\chapter*{Conclusion}

The performance of a heterogeneous processor such as the Zynq is influenced by a number of factors. The first one being the way the memory is implemented. The use of buffers increases the computational intensity, moving the performance of the system further to the right and closer to the peak computational performance. Another factor that plays an important role in the memory architecture is the AXI interconnect that is being used and the amount being used. The AXI HP interconnect has the highest performance for large datasets. The Zynq has 4 on board and only if All 4 are employed at the same time the DDR memory bandwidth becomes the bottleneck. As the TRD shows using the 4 AXI HP ports purely for performing the required action is an unlikely scenario. Using a High level synthesis tool presents a serious advantage in programmer productivity. HLS tools such as Vivado HLS give meaningful reports after synthesis and enable a designer to act upon this information. This is in stark contrast to having to synthesize the complete system, which can take up to an hour for a design the size of the TRD.\\
The second factor influencing the performance are the directives applied to the system. As the test shows these can have an immense impact on the performance of a system. Using the solution feature of Vivado the effect of different directives can also be measured in mere minutes. Some of these directives, such as loop unrolling, can influence the computational intensity of the system. Most directives however don't have this effect and can be represented in the roofline models as ceilings for the computational performance.\\
A third factor influencing the performance of the system is the resource consumption. In the case of the TRD a number of IP cores are necessary to transform, read and write the data. Also for every duplicate of the core performing the required action a DMA controller is necessary. Finally due to the low amount of AXI ports these will usually be the constraining factor in scaling the system.\\
The Zynq SoC presents an interesting hybrid between a CPU and an FPGA with serious performance, especially when considered in an embedded system context. Pairing this with a powerful HLS tool allows the designer to get performance from this system relatively quickly.